\documentclass{article}
\author{Dr. Yu-Chi "Larry" Ho}
\title{Life of an Academic in the US\footnote{Edited by Chang Liu}}

\begin{document}
\maketitle

Except for three years of full time work in industry I have spent my life in academia in  the US.  Even though I had only worked for one university in my entire career, I did spent semester long sabbaticals in residence in UC Berkeley, U.Texas Autstin, Imperial College in London and short visits to countless scholarly institutions around the world. Thus, it might be worthwhile for me to describe for my Chinese sciencenet readers my 40+ years experience and contrast them wherever appropriate with practices in China. Of course, the types of universities in the US are enormous. I am talking primarily of research universities. Excellent purely teaching college, such Williams College in Wiliamstown, Massachusetts, Swarthmore College in Philadelphia, Pa, etc. are not included. Lastly, what I describe below are my personal experiences and should not be construed as typical or general truth.


To advance your career in a first class university, you have basically two routes.  First you can be a great teacher (not just good teacher). This means you have to be able to create, organize and teach large and popular undergraduate courses such as beginning economics, physics, or computer science. This is easily said than done. Because the number of people in a department technically qualified to teach such course are many. You have to demonstrate extraordinary teaching and innovative ability. Furthermore, the number of such course with large undergraduate demand in a university are few. In my 40+ years at Harvard, I have seen many very good or near great teachers did not get tenure and had to leave. (footnote: In the US, we have the so-called seven year rule: you either get tenure or you must leave the university within a seven year trial period. Universities do not observe this rule will be blacklisted by the American Association of University Professors AAUP.). The other route to advancement is of course through research. While teaching is basically a local phenomenon, research is world wide. The cost benefit of excellent research is greatly in favor of research over teaching. But here in first rank universities, you are competing with the entire world. For example in tenure decisions at Harvard, comparison with other authorities in your field world wide are deliberately sought when your case comes up for evaluation. After my own promotion from within Harvard in 1965, our department did not promote another junior faculty member to tenure for 30 years in the systems field. Although you will not get the president and deans to publicly admit this, I have recognized early and have continued to counsel young faculty members that ``Harvard pays you to maintain and enhance her international reputations. Everything else is secondary''.  Thus, rule \#1 – Establish your world wide scholarly reputation early.


To pursue research you must have funds and students. In fact without funding you cannot support and thus have students. Depending on the availability of scholarships or fellowships, most graduate students are supported by the research funds of their thesis advisers in the US. It cost US\$50,000 per year more or less to support one graduate student. ( note added June 23, 2013: Because of inflation and other factors, the cost has now risen to 65-70,000 per year)  If you have 5-6 ph.d. students  in the pipeline at any one time, then your minimal budget is more than quarter a million dollars per year which the professor must apply for and compete with other scientists in your field nation wide.  In the first two years of supporting a graduate student,  you basically get no return. A ph.d student only begins to produce in the third and fourth year. Thus, a professor invests quite a bit when s/he begins to support a student.  I have known colleagues who swear that s/he will never admit another Chinese students because they come for one or two years and then leave for a better school or opportunity. While America is a free country, legally there is no obligation to stay and complete your ph.d, ethically one has certain obligations after a professor has invested so much in you.  This is something not all Chinese students realize and such behavior burns the bridge (过河折桥) for student that plans to follow. 


At least 25\% of you working time for active academic scientists are consumed by writing proposals, reports, and papers in addition to real research effort. Thus, rule \#2 – Learn to write and speak well and know what subjects are hot, i.e.environment makes the hero (Of course, even better is to create a research subject yourself and convince the world it is hot. environment makes the hero).


Both rules \#1 and \#2 mean that you must be “visible” to the rest of the world. You do this by publishing papers in prestige journals and give good talks at conferences. Many scientists consider doing the research is of paramount importance, but writing a paper and giving a talk on your results as more or less trivial tasks in comparison. But unless you have truly world shaking results such as the theory of relativity or the human genetic code, your work will be competing for attention with thousands of others who are just as smart as your are. In fact, it is my personal opinion that having a good idea, writing a good paper to report it, and giving a good presentation of it are three SEPARATE, and equally IMPORTANT endeavors. Efforts involved in each are distinct and different. Giving a presentation does not mean producing a set of PowerPoint slides  from cut-and-pasting your written papers and nor reading paragraphs from your papers. Too often we see otherwise brilliant scientists give poor or un-intelligible talks.  In fact, by definition a good talk must be understandable to the average audience and yet at the same time be impressive to the experts.  Statistics say that an average published technical paper is read by less than five persons including the editor and referees. But a good talk are heard by dozens of people and hundred or even thousands for plenary talks.  One month after your talk most of the audience will not remember what you talked about without reinforcement. But they will remember the fact that you gave a good talk for many years. The reward of such audience-centered presentation effort often surprises you in unexpected ways. On this point, the program officers of various government funding agencies go to conferences primarily for the purpose of finding out what topics are hot and who are worth supporting. They are often in the audiences when you give a talk. The importance of giving a good and understandable presentation are obvious.  Yet I am continuously surprised to find brilliant scientists insist on giving self-centered, incomprehensible, and arrogant talks. When I was younger, I use to consider myself ignorant if I cannot follow a talk. Nowadays, if I cannot understand what the speaker is saying, I blame the speaker. It is his duty to make his presentation clear and not wasting my time. My motto is “one can make anything understandable to anybody in any amount of time at the appropriate level”.


Traditionally, devoting your life to scholarly endeavor means committing to relative poverty since academic salary cannot compare with that of industry and commerce. However, in science and technology this is not necessarily true anymore. Consulting for industry can add to your income substantially. But the real benefit of consulting is the fact you are dealing with real problems which often inspire new research directions. Any success you achieve will have a cheering team ( 拉拉队) automatically built in. No extra efforts are needed to convince others of the importance of your work. And you will be not spending effort on something no one is interested in (钻入牛角尖).  Almost all US universities recognize this benefit and permit their faculty one day per week to pursue such consulting endeavors. In my own case, I have continuously consulted throughout my academic career and can claim all my good research ideas came from taking on consulting work which I know very little about at the start. The right kind of consulting is truly a win-win-win situation for the client, the university, and yourself.


Another fringe benefit of academic life is that you get to travel extensively by attending international conferences on official business. Unlike business meetings where time is tight and you basically go in and out of a place with little time for leisure activity.  You generally find time during or after conferences for tourist activities. (see the example of the blog by Dr. 王鸿飞飞  recently in these pages).  The marginal cost to you is minimal compare to the entire cost of travel. There is a travel book entitled “1000 place to visit before you die”. In 40+ years of academic travel, my wife and I have been to 230 of the 1000 places listed as well as places not listed. There is an old Chinese saying: 走千里路胜读万卷書. You actually learn a lot from being a tourist.  Furthermore, in academia you have friends all over the world that you have only met for the first time. Because of your shared interest in the subject of research, it is truly 一见如古. They are excellent local guides. (footnote: I once asked one of my Chinese graduate student to help a visitor from South America. Upon return, she was so excited to tell me that this complete stranger understood the exact subtleties of her thesis research despite the fact they have never met, are from different countries separated by geography and generation. Such meeting of minds is the indescribable joys of research). 


Speaking about the joys of research, when you discovered something good for the first time after months of hard work you enjoy that few seconds of ecstasy that you are the only person in the world who knows this truth. Such a feeling is difficult to describe. You will not be able to sit still, you pace back and forth, you will not be able to sleep that night, and sometimes your stomach will tie into knots in pleasurable pain. Although I cannot say for sure since I don’t have the experience of commerce. This pleasure of discovery cannot be less than if you just win a big contract or made a killing in the stock market. If I experience such feeling once a year, I am very satisfied. If one looks back over his life on all the papers he has written and published, perhaps 10% stood the test of time. This is very much like travel. You are glad you have been to various places. But only a few localities are truly memorable.



Above all, the life of an academic is basically flexible and on your terms. You are your own boss. With the exception of funding, you have far less of the many mundane and uninteresting duties of an ordinary business person. Nearly 50% of your time you can devote to things you truly enjoy, i.e., research and seeking truth. And I always tell my students that in life if you like 50% of the things you do in your job, it is a GREAT job! Because you like what you do, you do work very hard. Weekends and holidays are no different. 20 hour day and 100 hour week are frequent. Even at my age and officially retired , more than half of my waking hours are dealing with scholarly and technical matters. 



Things I learned through many stays in China and articles I read here in this sciencenet blog tell me that the academic life in China is beginning to globalize and resemble academic life in rest of the world. I look forward to exchange more notes with colleagues here and elsewhere  in the future.

\end{document}
