\documentclass{article}
%\usepackage{}
\begin{document}

\section{AAA}
A reference to this subsection
\label{sec:this} looks like:
‘‘see section~\ref{sec:this} on
page~\pageref{sec:this}.’’


Now that we define a new label \label{name}


\newpage

\section{BBB}
a reference is pointed to ~\ref{sec:this} OK?

Here is the name area ~\ref{name}

\newpage


\section{CCC}

Now that we use the ref's page as the index, which is ~\pageref{sec:this}



\section{DDD}

Footnote is where it lays at the bottom of the page\footnote{This is a footnote for the page}


\section{EEE}

Example:

this is a print for the underline \underline{HaHaHaHaHaHaHaH}

this is for leaning \emph{lean}

\section{FFFF: List Example}

\flushleft

\begin{enumerate} % enumerate用于带序号的列表 
\item  You can begin in this items
\begin{itemize} % 用于简单列表
\item But it might start to look silly.
\item[-] with a dash
\item new item, 3th one.
\end{itemize}

\item This is a second item
\begin{description}
\item[Stupid] THis is stupid
\item[Smart] Hello, world
\item[Honest] Hello, honest
\end{description}

\end{enumerate}


\section{GGG}
A typographical rule of thumb
for the line length is:
\begin{quote}
On average, no line should
be longer than 66 characters.
\end{quote}
This is why \LaTeX{} pages have
such large borders by default and
also why multicolumn print is
used in newspapers.\newline

I know only one English poem by
heart. It is about Humpty Dumpty.
\begin{flushleft}
\begin{verse}
Humpty Dumpty sat on a wall:\\
Humpty Dumpty had a great fall.\\
All the King’s horses and all
the King’s men\\
Couldn’t put Humpty together
again.
\end{verse}
\end{flushleft}



\section{HHH}

between the verbatim and its ending, all texts are printed out, including the blank and separate line, without any \LaTeX{} command.
similarly, using \\{verb} is the same.

\begin{verbatim}
10 PRINT "HELLO WORLD";
20 GOTO 10
\end{verbatim}

% verbatim* will output all the result in this pharase
\begin{verbatim*}
the starred version of
the    verbatim environment emphasizes
the spaces   in the text
\end{verbatim*}


\section{III: table}

% l for left column, r for right column, c for middle column, p{w} for width, | for vertical
%% in tabular environment, & jumps to next column, \\ jumps to new line, \hline insert horizantal line, \cline{j-i} insert column jth to ith

\begin{tabular}{|r|l|} % two columns, right-aligned and left-aligned
\hline
7C0 & hexadecimal \\
3700 & octal \\ \cline{2-2}
111110000 & binary \\
\hline

\hline
1984 & decimal \\
\hline


\hline
\end{tabular}
\newline

\begin{tabular}{|p{4.7cm}|}
\hline
Welcome to Boxy’s paragraph.
We sincerely hope you’ll
all enjoy the show.\\
\hline
\end{tabular}




\end{document}

